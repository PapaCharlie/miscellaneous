\section{Exporting values using JSON.}
Liquidsoap can export any language value in JSON using
\verb+json_of+.

The format is the following :

\begin{itemize}
\item \verb+() : unit+ -> \verb+null+
\item \verb+true: bool+ -> \verb+true+
\item \verb+"abc" : string+ -> \verb+"abc"+
\item \verb+23 : int+ -> \verb+23+
\item \verb+2.0 : float+ -> \verb+2.0+
\item \verb+[2,3,4] : [int]+ -> \verb+[2,3,4]+
\item \verb+[("f",1),("b",4)] : [(string*int)]+ -> \verb+{"f": 1, "b": 4}+
\item \verb+("foo",123) : string*int+ -> \verb+[ "foo", 123 ]+
\item \verb+s : source+ -> \verb+"<source>"+
\item \verb+r : int ref+ -> \verb+{"reference": 4}+
\item \verb+%mp3 : encoder+ -> \begin{verbatim}
"%mp3(stereo,bitrate=128,samplerate=44100)" }
\end{verbatim}

\item \verb+r : request+ -> \verb+"<request>"+
\item \verb+f : 'a -> 'b+ -> \verb+"<fun>"+

\end{itemize}
The two particular cases are:

\begin{itemize}
\item Products are exported as lists
\item Lists of type \verb+string*'a+ are exported as object of the form \verb+{"key": value}+ 

\end{itemize}
Output format is pretty printed by default. A compact output can
be obtained by using the optional argument: \verb+compact=true+.

\section{Importing values using JSON.}
If compiled with json-wheel support, Liquidsoap can also
parse JSON data into values. using \verb+of_json+.

The format is a subset of the format of exported values with the notable
difference that only ground types (\verb+int+, \verb+floats+, \verb+string+, ...)
are supported and not variable references, sources, formats,
requests and functions:

\begin{itemize}
\item \verb+null+ -> \verb+() : unit+
\item \verb+true/false+ -> \verb+true/false : bool+
\item \verb+"abc"+ -> \verb+"abc" : string+
\item \verb+23+ -> \verb+23 : int+
\item \verb+2.0+ -> \verb+2.0 : float+
\item \verb+[2,3,4]+ -> \verb+[2,3,4] : int+
\item \verb+{"f": 1, "b": 4}+ -> \verb+[("f",1),("b",4)] : [(string*int)]+
\item \verb+[ "foo", 123 ]+ -> \verb+("foo",123) : string*int+

\end{itemize}
The JSON standards specify that a proper JSON payload can only be an array or an
object. However, simple integers, floats, strings and null values are
also accepted by Liquidsoap.

The function \verb+of_json+ has the following type:

\begin{verbatim}
  (default:'a,string)->'a
\end{verbatim}
The default parameter is very important in order to assure 
type inference of the parsed value. Its value constrains
the parser to only recognize JSON data of the the default value's 
type and is returned in case parsing fails.

Suppose that we want to receive a list of metadata, encoded as an object:

\begin{verbatim}
{ "title": "foo",
 "artist": "bar" }
\end{verbatim}
Then, you would use of\_json with default value \verb+[("error","fail")]+ and do:

\begin{verbatim}
# Parse metadata from json
m = of_json(default= [("error","fail")], json_string)
\end{verbatim}
The type of the default value constrains the parser. For instance, in the 
above example, a JSON string \verb+"[1,2,3,4]"+ will not be accepted and the 
function will return the values passed as default.

You can use the default value in two different ways:

\begin{itemize}
\item To detect that the received json string was invalid/could not be parsed to the expected type. In the example above, if \verb+of_json+ return a metadata value of \verb+[("error","fail")]+ (the default) then you can detect in your code that parsing has failed.
\item As a default value for the rest of the script, if you do not want to care about parsing errors.. This can be useful for instance for JSON-RPC notifications, which should not send any response to the client anyway.

\end{itemize}
