\section{Streaming to Shoutcast}
Although Liquidsoap is primarily aimed at streaming to Icecast servers (that provide 
much more features than Shoutcast), it is also able to stream to Shoutcast.

\subsection{Shoutcast output}
Shoutcast server accept streams encoded with the MP3 or AAC/AAC+ codec. You to compile Liquidsoap with 
\verb+lame+ support, so it can encode in MP3. Liquidsoap also has support for AAC+ encoding
using libaacplus or using an \href{external_encoders.html}{external encoder}. The recommended format is MP3.

Shoutcast output are done using the \verb+output.shoutcast+ operator with the appropriate parameters.
An example is:

\begin{verbatim}
source = single("audiofile.ogg")

output.shoutcast(%mp3, host="shoutcast.example.org",
                     port=8000, password="changeme",
                     source)
\end{verbatim}
As usual, \verb+liquidsoap -h output.shoutcast+ gives you the full list of options for this operator.

\subsection{Shoutcast as relay}
A side note for those of you who feel they ``need'' to use Shoutcast for non-technical reasons (such as their stream 
directory service...): you can still benefit from Icecast's power by streaming to an Icecast server, and then relaying 
it through a shoutcast server.

In order to do that, you have to alias the root mountpoint (``\verb+/+'') to your MP3 mountpoint in your icecast server 
configuration, like this:

\begin{verbatim}

&lt;alias source="/" dest="/mystream.mp3" /&gt;
\end{verbatim}
Be careful that icecast often aliases the status page (\verb+/status.xsl+) with the \verb+/+. In this case, comment 
out the status page alias before inserting yours.

