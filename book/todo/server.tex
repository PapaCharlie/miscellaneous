\section{Interaction with the server}
Liquidsoap starts with one or several scripts as its configuration,
and then streams forever if everything goes well.
Once started, you can still interact with it by means of the \emph{server}.
The server allows you to run commands. Some are general and always available,
some belong to a specific operator. For example the \verb+request.queue()+ instances register commands to enqueue new requests, the outputs register commands
to start or stop the outputting, display the last ten metadata chunks, etc.

The protocol of the server is a simple human-readable one.
Currently it does not have any kind of authentication and permissions.
It is currently available via two media: TCP and Unix sockets.
The TCP socket provides a simple telnet-like interface, available only on
the local host by default.
The Unix socket interface (\emph{cf.} the \verb+server.socket+ setting)
is through some sort of virtual file.
This is more constraining, which allows one to restrict the use of the socket
to some priviledged users.

You can find more details on how to configure the server in the
\href{help.html#settings}{documentation} of the settings key \verb+server+,
in particular \verb+server.telnet+ for the TCP interface and \verb+server.socket+
for the Unix interface.
Liquidsoap also embeds some \href{help.html#server}{documentation}
about the available server commands.

Now, we shall simply enable the Telnet interface to the server,
by setting \verb+set("server.telnet",true)+ or simply passing the \verb+-t+ option on
the command-line.
In a \href{complete_case.html}{complete case analysis} we set up a \verb+request.queue()+
instance to play user requests. It had the identifier \verb+"queue"+.
We are now going to interact via the server to push requests into that queue:

\begin{verbatim}
dbaelde@selassie:~$ telnet localhost 1234
Trying 127.0.0.1...
Connected to localhost.localdomain.
Escape character is '^]'.
queue.push /path/to/some/file.ogg
5
END
request.metadata 5
[...]
END
queue.push http://remote/audio.ogg
6
END
request.trace 6
[...see if the download started/succeeded...]
END
exit
\end{verbatim}
Of course, the server isn't very user-friendly.
But it is easy to write scripts to interact with Liquidsoap in that way,
to implement a website or an IRC interface to your radio.
However, this sort of tool is often bound to a specific usage, so we have
not released any of ours. Feel free to
\href{mailto:savonet-users@lists.sf.net}{ask the community}
about code that you could re-use.

