\section{Documentation index}
\textbf{How to use}: Start with the \href{quick_start.html}{quickstart} and 
make sure you learn \href{help.html}{how to find help}. Then it's as you like:
go for another \href{#general}{general tutorial}, or a 
\href{#specific}{specific example},
pick a \href{#core}{basic notion}, or some examples from the 
\href{cookbook.html}{cookbook}. If you've understood all you need,
just browse the \href{reference.html}{reference} and compose your dream stream.

If you downloaded a source tarball of liquidsoap, you may first read the
\href{build.html}{build instructions}.

\subsection{General tutorials}
\begin{itemize}
\item \href{quick_start.html}{Quickstart}: where anyone should start.
\item \href{complete_case.html}{Complete case analysis}: an example that is not a toy.
\item \href{advanced.html}{Advanced}: overview of more advanced features for serious usage.
\item \href{help.html}{How to find help} about operators, settings, server commands, etc.
\item \href{cookbook.html}{Cookbook}: contains lots of idiomatic examples.
\item \href{faq.html}{Frequently Asked Questions, Troubleshooting}

\end{itemize}
\subsection{Reference}
\begin{itemize}
\item \href{reference.html}{API reference}: All the builtin functions of liquidsoap.
\item \href{settings.html}{Settings}: The list of available settings for liquidsoap.
\item \href{language.html}{Script language}: A more detailed presentation.
\item \href{encoding_formats.html}{Encoding formats}: The available formats for encoding outputs.
\item \href{json.html}{JSON import/export}: Importing and exporting language values in JSON.
\item \href{ladspa.html}{LADSPA plugins}: Using LADSPA plugins.

\end{itemize}
\subsection{Core}
\begin{itemize}
\item Basic concepts: \href{sources.html}{sources}, \href{clocks.html}{clocks} and \href{requests.html}{requests}.
\item \href{stream_content.html}{Stream contents}: what kind of streams are supported, and how.
\item \href{script_loading.html}{Script loading}: load several scripts, learn about the script library.
\item \href{phases.html}{Execution phases}

\end{itemize}
\subsection{Specific tutorials }
\begin{itemize}
\item \href{blank.html}{Blank detection}
\item \href{metadata.html}{Customize metadata}
\item \href{seek.html}{Seek and cue support}: seek and set cue-in and cue-out points in sources.
\item \href{external_decoders.html}{External decoders}: use an external program for decoding audio files.
\item \href{external_streams.html}{External streams}: use an external program for streaming audio data.
\item \href{external_encoders.html}{External encoders}: use an external audio encoding program.
\item \href{http_input.html}{HTTP input}: relay external streams.
\item \href{harbor.html}{Harbor input}: receive streams from icecast and shoutcast source clients.
\item \href{harbor_http.html}{Interaction with the Harbor}: interact with a running liquidsoap using the Harbor server.
\item \href{server.html}{Interaction with the server} interact with a running liquidsoap instance using the telnet server.
\item \href{icy_metadata.html}{ICY metadata update}: manipulate and configure metadata update in Icecast.
\item \href{replay_gain.html}{Normalization and replay gain}: normalize audio data.
\item \href{request_sources.html}{Requests-based sources}: create advanced sources using requests.
\item \href{shoutcast.html}{Shoutcast output}: output to shoutcast.
\item \href{dynamic_sources.html}{Dynamic source creation}: dynamically create sources using server requests.
\item \href{smartcrossfade.html}{Smart crossfading}: define custom crossfade transitions.
\item \href{secure_server_access.html}{Secure command server}: secure authenticated access to the command server.
\item \href{in_production.html}{Using in production}: integrate liquidsoap scripts in a production environment.
\item \href{flows.html}{Liquid Flows}: add your radio to the \href{http://savonet.sourceforge.net/flows.html}{webpage} of proud users.

\end{itemize}
\subsection{User scripts}
\begin{itemize}
\item \href{bubble.html}{Bubble}: a simple example of a database interface as a custom protocol.
\item \href{geekradio.html}{Geekradio}
\item \href{radiopi.html}{RadioPi}
\item \href{dolebrai.html}{Dolebraï}
\item \href{kube.html}{Kube}
\item \href{frequence3.html}{Frequence3}
\item \href{radio-nova.html}{Listen to Radio Nova}
\item \href{video-static.html}{Video with a single static image}

\end{itemize}
\subsection{Code snippets}
\begin{itemize}
\item \href{scripts/index.html}{Code example index}

\end{itemize}
