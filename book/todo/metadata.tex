\section{Customize metadata using liquidsoap}
Liquidsoap has several mechanism for manipulating the metadata attached to your stream. In this page we quickly detail and compare the different operators, see the \href{reference.html}{language reference} for full details about them.

\textbf{Warning}. The protocol used by Shoutcast and Icecast before version 2 does not support many fields. It mainly support one: \verb+song+. So, if you need to customize the metadata displayed by these servers, you should customize only the \verb+song+ metadata.

\subsection{Rewrite metadata}
\verb+rewrite_metadata+ rewrites metadata using a list of (target,rules). The semantic for the replacement rules is that of the \verb+%+ function. Namely, \verb+(pattern % [...,(k,v),...])+ changes in \verb+pattern+ occurences of:

\begin{itemize}
\item \verb+'$(k)'+ into \verb+"v"+;
\item \verb+'$(if $(k2),"a","b")'+ into \verb+"a"+ if \verb+k2+ is found in the list, \verb+"b"+ otherwise.

\end{itemize}
A sample code using this operator can be:

\begin{verbatim}
# The 'display_artist' field is passed using annotate.
pattern = 
  '$(if $(display_artist),"$(display_artist)","$(artist)")'
rewrite_metadata([("artist",pattern)],source)
\end{verbatim}
\subsection{Map\_metadata }
The \verb+map_metadata+ operator applies a specified function to transform each metadata chunk of a stream. It can be used to add or decorate metadata, but is also useful in more complex cases.

A simple example using it is:

\begin{verbatim}
# A function applied to each metadata chunk
def append_title(m) =
  # Grab the current title
  title = m["title"]

  # Return a new title metadata
  [("title","#{title} - www.station.com")]
end

# Apply map_metadata to s using append_title
s = map_metadata(append_title, s)
\end{verbatim}
The effect of \verb+map_metadata+ by default is to update the metadata with the returned values. 
Hence in the function \verb+append_title+ defined in the code above returns a new metadata for the 
label \verb+title+ and the other metadata remain untouched. You can change this by using the 
\verb+update+ option, and you can also remove any metadata (even empty one) using the \verb+strip+ option.

See the documentation on \verb+map_metadata+ for more details.

A more complex example is the \verb+rewrite_metadata+ operator, which is implemented using \verb+map_metadata+ as follows:

\begin{verbatim}
# Rewrite metadata on the fly using a list of (target,rule).
# @category Source / Track Processing
# @param l  List of (target,value) rewriting rules
# @param ~insert_missing \
#           Treat track beginnings without metadata \
#           as having empty ones.
def rewrite_metadata(l,~insert_missing=true,s)
  def map(m)
    def apply(x)
      label = fst(x)
      meta = snd(x)
      if list.mem_assoc(label,l) then
        pattern = l[label]
        (label,pattern % m)
      else
        (label,meta)
      end
    end
    m = list.map(apply,m)
    def add(m,x)
      label = fst(x)
      pattern = snd(x)
      # If m does not have (label,_), then it was
      # not processed previously, we have to
      # add it now..
      if not list.mem_assoc(label,m) then
        list.append(m,[(label,pattern % m)])
      else
        m
      end
    end
    list.fold(add,m,l)
  end
  map_metadata(map,insert_missing=insert_missing,s)
end
\end{verbatim}
\subsection{Insert\_medatata}
This operator is used for inserting metadata using a server command. If you have an \verb+insert_metadata+ node named \verb+ID+ in your configuration, you can connect to the server (either telnet or socket) and execute commands like: 

\begin{verbatim}
ID.insert key1="val1",key2="val2",...
\end{verbatim}
