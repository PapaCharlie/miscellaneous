\chapter{Frequently asked questions}
\subsection{What does this message means?}
\subsubsection{Type error}
Liquidsoap might also reject a script with a series of errors of the form
\begin{verbatim}
this value has type ... but it should be a subtype of ...
\end{verbatim}
Usually the last error tells you what the problem is, but the previous errors
might provide a better information as to where the error comes from.

For example, the error might indicate that a value of type \verb+int+ has been
passed where a float was expected, in which case you should use a conversion, or
more likely change an integer value such as \verb+13+ into a float \verb+13.+.

A type error can also show that you're trying to use a source of a certain
content type (e.g., audio) in a place where another content type (e.g., pure
video) is required. In that case the last error in the list is not the most
useful one, but you will read something like this above:

\begin{verbatim}
At ...:
  this value has type
    source(audio=?A+1,video=0,midi=0)
    where ?A is a fixed arity type
  but it should be a subtype of
    source(audio=0,video=1,midi=0)
\end{verbatim}
Sometimes, the type error actually indicates a mistake in the order or labels of
arguments. For example, given \verb+output.icecast(mount="foo.ogg",source)+
liquidsoap will complain that the second argument is a source
(\verb+source(?A)+) but should be a format (\verb+format(?A)+): indeed, the
first unlabelled argument is expected to be the encoding format, e.g.,
\verb+%vorbis+, and the source comes only second.

Finally, a type error can indicate that you have forgotten to pass a mandatory
parameter to some function. For example, on the code
\verb+fallback([crossfade(x),...])+, liquidsoap will complain as follows:

\begin{verbatim}
At line ...:
  this value has type
    (?id:string, ~start_next:float, ~fade_in:float,
     ~fade_out:float)->source(audio=?A,video=?B,midi=0)
    where ?B, ?A is a fixed arity type
  but it should be a subtype of
    source(audio=?A,video=?B,midi=0)
    where ?B, ?A is a fixed arity type
\end{verbatim}
Indeed, \verb+fallback+ expects a source, but \verb+crossfade(x)+ is still a
function expecting the parameters \verb+start_next+, \verb+fade_in+ and
\verb+fade_out+.

\subsubsection{That source is fallible!}
See the \href{quick_start.html}{quickstart}, or read more about
\href{sources.html}{sources}.

\subsubsection{Clock error}
Read about \href{clocks.html}{clocks} for the errors
\verb+a source cannot belong to two clocks+ and
\verb+cannot unify two nested clocks+.

\subsubsection{We must catchup x.xx!}
This error means that a clock is getting late in liquidsoap. This can be caused
by an overloaded CPU, if your script is doing too much encoding or processing:
in that case, you should reduce the load on your machine or simplify your
liquidsoap script. The latency may also be caused by some lag, for example a
network lag will cause the icecast output to hang, making the clock late.

The first kind of latency is problematic because it tends to accumulate,
eventually leading to the restarting of outputs:
\begin{verbatim}
Too much latency!  Resetting active source...
\end{verbatim}

The second kind of latency can often be ignored: if you are streaming to an
icecast server, there are several buffers between you and your listeners which
make this problem invisible to them. But in more realtime applications, even
small lags will result in glitches.

In some situations, it is possible to isolate some parts of a script from the
latency caused by other parts. For example, it is possible to produce a clean
script and back it up into a file, independently of its output to icecast (which
again is sensitive to network lags).  For more details on those techniques, read
about \href{clocks.html}{clocks}.

\subsubsection{Unable to decode ``file'' as @{audio=2;video=0;midi=0}@!}
This log message informs you that liquidsoap failed to decode a file, not
necessarily because it cannot handle the file, but also possibly because the
file does not contain the expected media type. For example, if video is
expected, an audio file will be rejected.

The case of mono files is often surprising. Since liquidsoap does not implicitly
convert between media formats, input files must be stereo if the output expects
stereo data. As a result, people often get this error message on files which
they expected to play correctly. The simple way to fix this is to use the
\verb+audio_to_stereo()+ operator to allow any kind of audio on its input, and
produce stereo as expected on its output.

\subsubsection{Exceptions}
Liquidsoap dies with messages such as these by the end of the log:

\begin{verbatim}
... [threads:1] Thread "XXX" aborts with exception YYY!
... [stderr:3] Thread 2 killed on uncaught exception YYY.
... [stderr:3] Raised at file ..., line ..., etc.
\end{verbatim}
Those internal errors can be of two sorts:

\begin{itemize}
\item \textbf{Bug}: Normally, this means that you've found a bug, which you
  should report on the mailing list or bug tracker.
\item \textbf{User error}: In some cases, we let an exception go on user errors,
  instead of nicely reporting and handling it. By looking at the surrounding log
  messages, you might realize that liquidsoap crashed for a good reason, that
  you are responsible for fixing. You can still report a bug: you should not
  have seen an exception and its backtrace.

\end{itemize}
In any case, once that kind of error happens, there is no way for the user to
prevent liquidsoap from crashing. Those exceptions cannot be caught or handled
in any way at the level of liquidsoap scripts.

\subsection{Troubleshooting}
\subsubsection{Pulseaudio}
When using ALSA input or output or, more generaly any audio input or output that
is not using pulseaudio, you should disable pulseaudio, which is often installed
by default. Pulseaudio emulates ALSA but this also generates bugs, in particular
errors of this form:

\begin{verbatim}
Alsa.Unknown_error(1073697252)!
\end{verbatim}
There are two things you may do:

\begin{itemize}
\item Make sure your alsa input/output does not use pulseaudio
\item Disable pulseaudio on your system

\end{itemize}
In the first case, you should first find out which sound card you want to use,
with the command \verb+aplay -l+. An example of its output is:

\begin{verbatim}
**** List of PLAYBACK Hardware Devices ****
card 0: Intel [HDA Intel], device 0: STAC92xx Analog [STAC92xx Analog]
  Subdevices: 1/1
  Subdevice #0: subdevice #0
\end{verbatim}
In this case, the card we want to use is: device \verb+0+, subdevice \verb+0+,
thus: \verb+hw:0,0+. We now create a file \verb+/etc/asound.conf+ (or
\verb+~/.asoundrc+ for single-user configuration) that contains the following:

\begin{verbatim}
pcm.liquidsoap {
        type plug
        slave { pcm "hw:0,0" }
}
\end{verbatim}
This creates a new alsa device that you can use with liquidsoap. The \verb+plug+
operator in ALSA is used to work-around any hardward limitations in your device
(mixing multiple outputs, resampling etc.). In some cases you may need to read
more about ALSA and define your own PCM device.

Once you have created this device, you can use it in liquidsoap as follows:
\begin{verbatim}
input.alsa(device="pcm.liquidsoap", ...)
\end{verbatim}
In the second case -- disabling pulseaudio, you can edit the file
\verb+/etc/pulse/client.conf+ and change or add this line:

\begin{verbatim}
autospawn = no
\end{verbatim}
And kill any running pulseaudio process:

\begin{verbatim}
killall pulseaudio
\end{verbatim}
Otherwise you may simply remove pulseaudio's packages, if you use Debian or
Ubuntu:

\begin{verbatim}
apt-get remove pulseaudio libasound2-plugins
\end{verbatim}

\subsubsection{Listeners are disconnected at the end of every track}
Several media players, including renowned ones, do not properly support
Ogg/Vorbis streams: they treat the end of a track as an end of file, resulting
in the disconnection.

Players that are affected by this problem include VLC.  Players that are not
affected include ogg123, liquidsoap.

One way to work around this problem is to not use Ogg/Vorbis (which we do not
recommend) or to not produce tracks within a Vorbis stream.  This is done by
merging liquidsoap tracks (for example using
\verb+add(normalize=false,[blank(),source])+) and also not passing any metadata
(which is also a result of the previous snippet).

\subsubsection{Encoding blank}
Encoding pure silence is often too effective for streaming: data is so
compressed that there is nothing to send to listeners, whose clients eventually
disconnect. Therefore, it is a good idea to use a non-silent jingle instead of
\verb+blank()+ to fill in the blank. You can also achieve various effects using
synthesis sources such as \verb+noise()+, \verb+sine()+, etc.
